%!TEX root = ../artigo.tex
\section{Resumo} % (fold)
\label{sec:resumo}
A análise léxica é uma parte importante nas áreas de compiladores e linguagens de programação. Através da análise léxica é possível verificar se um determinado símbolo pertence à um alfabeto. Além disso, o analisador léxico - programa responsável pela análise léxica - compoe a primeira fase do compilador em que lê o código em texto, identifica os caracteres e agrupa-os numa sequencia de símbolos ou tokens.
O analisador léxico recebe como entrada um conjunto de caracteres. Quando encontra um lexeme - sequência de caracteres que corresponde à um padrão especificado na linguagem ou uma ocorrência de um token - insere numa tabela de símbolos que posteriormente será utilizada por outras etapas do compilador. Assim o analisador léxico produz como saída uma sequencia de palavras (tokens) que representa um tipo de unidade léxica.\cite{aho2007compilers}\cite{Sebesta201201}

Neste trabalho apresentamos ferramentas didáticas que ajudam no aprendizado da análise léxica, as seguintes ferramentas serão abordadas: C-gen, Sintelo e Verto.

% section resumo (end)